\documentclass[a4paper,12pt]{book}
\usepackage{mathptmx}
\usepackage{hyperref}
\usepackage{cancel}
\usepackage{amsmath}
\usepackage{amssymb}
\usepackage{graphicx}
\usepackage{pgfplots}
\usepackage{tikz}
\usepackage{xcolor}
\usepackage{tcolorbox}
\newtheorem{definition}{Definition}

\author{Hadhzy Organisation}
\title{Calculus Notes}
\date{\today}

% Define a new command for highlighted math
\newcommand{\highlightmath}[1]{\begin{tcolorbox}[colback=yellow!10!white, colframe=red!50!black, boxrule=0.5mm]#1\end{tcolorbox}}

\begin{document}
\maketitle

\tableofcontents

\chapter{Limits and continuity}
Limits are values that a function approaches as it goes to a certain point. 

There are numerous ways to find limits: direct substitution, factor and reduce or multiply by the conjugate, and L'Hopital's rule, limits can be also determinated graphically.

\textbf{Note}: Approaching a value doesn't mean reaching it. It's fine if the function doesn't reach the value, as long as it gets close to it.

\highlightmath{
\[
\lim_{x \to 2} \frac{x}{x^2}
\]
}
\textbf{Read as}: The limit as \( x \) approaches 2 of \( x \) over \( x \) squared.

\section{Continuity}

\section{One sided limits}

\section{Examples}

\chapter{Derivatives}

\chapter{Related Rates}

\chapter{Indefinite Integrals}

\section{Reversing the chain rule}

\chapter{The Definite Integral}

\section{Integrating with respect to y}

\section{Odd and even}
\section{Area between curves}
\section{U substitution}

\section{Integration with respect to y}
\chapter{Application of the definite integral}
\chapter{Trigonometry}

\highlightmath{
\[
let f(x) \text{ be a function defined on a closed interval} [a, b]
\]
}
\chapter{Summation}


\chapter{Series and Sequences}

\chapter{Complex Numbers}
Complex numbers are in the form of \( a + bi \), where \( a \) and \( b \) are real numbers and \( i \) is the imaginary unit.
i.e. \( i = \sqrt{-1} \)

\(i^2 = -1\)

\section{General form and polar form of complex numbers}
The \textbf(polar form of a complex number) is \(r \cos (\theta) + i \sin(\theta)\)

General form \(\mathbb{Z} = a + bi\)

\section{Complex conjugate}
\section(Argand diagram)

\section{Complex numbers as roots of cubic equations}

\section{Demoivre's theorem}

\section{Proof of Demoivre's theorem}

\chapter{Squeeze Theorem}
\chapter{Limits of Finite Sums}

\chapter{Geometry}


\end{document}
